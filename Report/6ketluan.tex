\section{Kết luận}

Sau quá trình tìm hiểu, thiết kế và thực hiện, nhóm đã hoàn thành đề tài “Hệ thống điều khiển đèn giao thông sử dụng vi điều khiển PIC16F887”. Hệ thống đã được lập trình đầy đủ các chức năng, bao gồm ba chế độ hoạt động: \textit{tự động}, \textit{thủ công}, và \textit{ban đêm}. Thông qua việc mô phỏng trên phần mềm Proteus, nhóm đã kiểm tra được tính đúng đắn của thuật toán điều khiển cũng như hoạt động phối hợp của các phần tử trong hệ thống.

Thông qua đề tài, nhóm đã củng cố và nâng cao các kiến thức về:
\begin{itemize}
    \item Lập trình vi điều khiển PIC bằng ngôn ngữ C.
    \item Thiết kế mạch nguyên lý và mạch điều khiển sử dụng LED, LCD, IC dịch 74HC595.
    \item Kỹ năng mô phỏng hệ thống điện tử với Proteus.
    \item Tư duy xử lý tín hiệu, thời gian và chế độ hoạt động trong hệ thống điều khiển thực tế.
\end{itemize}

Bên cạnh những kết quả đạt được, nhóm cũng nhận ra một số điểm cần cải thiện, chẳng hạn như tối ưu hóa mã nguồn, nâng cao tính thẩm mỹ khi mô phỏng, và tăng cường tính thực tế nếu triển khai hệ thống ngoài thực tế (ví dụ: sử dụng cảm biến phát hiện phương tiện, tích hợp RTC, hoặc giao tiếp không dây).

Qua bài tập lớn này, nhóm không chỉ có thêm kiến thức thực tiễn trong lĩnh vực vi điều khiển mà còn rèn luyện được kỹ năng làm việc nhóm, tư duy phân tích hệ thống và cách thức xây dựng một dự án kỹ thuật từ khâu thiết kế đến kiểm thử. Đây là tiền đề quan trọng cho các môn học, đồ án chuyên ngành và công việc kỹ sư sau này.
\cleardoublepage
